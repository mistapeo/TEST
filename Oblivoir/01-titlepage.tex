\documentclass[a4paper,hcr]{oblivoir}



%% title page
% \newcommand\MakeTitle{%
%  \begin{titlingpage}
%  \setcounter{page}{-1}%
%  \begin{adjustwidth*}{0mm}{-55mm}
%  \newlength\tmplen\setlength\tmplen{\textwidth}\addtolength\tmplen{60mm}
%  \fbox{%
%    \begin{minipage}{\tmplen}
%    \vspace*{90mm}
%    \begin{center}
%      \LARGE\bfseries\thetitle \\ \vskip\onelineskip
%      \normalfont\normalsize\theauthor
%    \end{center}
%    \vspace*{100mm}
%  \end{minipage}}
%  \end{adjustwidth*}
%  \end{titlingpage}
% }

\newcommand{\colorRule}[3][black]{\textcolor[HTML]{#1}{\rule{#2}{#3}}}
\newcommand\MakeTitle{%}
\begin{titlingpage}
\begin{flushleft}
\noindent
\\[-1em]
\color[HTML]{7137C8}
\makebox[0pt][l]{\colorRule[7137C8]{1.3\textwidth}{2pt}}
\par
\noindent

{
  \setstretch{1.4}
  \vfill
  \noindent {\huge \textbf{\textsf{\thetitle}}}
    \vskip 1em
  {\Large \textsf{\theauthor}}
    \vskip 2em
  \noindent {\Large \textsf{Data Strategy Division}}
  \vfill
}

\noindent
\includegraphics[width=30mm, left]{TG360\_logo.png}

\textsf{\thedate}
\end{flushleft}
\end{titlingpage}
}
% 폰트 설정
\ifPDFTeX
	\usepackage{mathpazo}
\else\ifLuaOrXeTeX
	\setmainfont{TeX Gyre Pagella}
 	\setsansfont[Scale=.95]{TeX Gyre Heros}
%% \setkomain/sansfont : see oblivoir-simpledoc.
	\setkomainfont(HCRBatang)(*-Bold)(UnGraphic)
	\setkosansfont[NanumGothic]()[](HCRDotum)
%% 수학 폰트
%	\usepackage{unicode-math}
%	\setmathfont{Asana-Math.otf}
\fi\fi

\ifLuaTeX
\def\interHANGUL{InterHangul}
\else\ifXeTeX
\def\interHANGUL{interhchar}
\fi\fi

\usepackage[mono=false]{libertine}

\usepackage{jiwonlipsum}

\pagestyle{hangul}

\usepackage{tcolorbox}

\begin{document}

\title{\texttt{jiwonlipsum} 패키지}
\author{Nova De Hi and Progress}
\date{v0.6, 2019/11/30}

\MakeTitle

\begin{abstract}
\texttt{lipsum}이라는 패키지가 있습니다. 외국에서 출판이나 그래픽 디자인, 편집 디자인, 폰트 디자인 따위의 작업을 하면서 텍스트 시안(sample)을 작성할 때, 의미 없는 문장을 채워넣기 위해 사용하는 것으로 대략 이해하시면 됩니다.  \texttt{jiwonlipsum}은 \texttt{lipsum}의 한글판이라 생각하시면 됩니다. 연암 박지원 선생의 \ccnm{열하일기} 가운데 `하룻밤에 아홉 번 강을 건너다(一夜九渡河記)' 문장을 빌어왔습니다. 패키지는 Nova De Hi님이 작성하였고 이 매뉴얼은 Progress가 썼습니다.
\end{abstract}

\tableofcontents*

\section{사용하기}

\begin{boxedverbatim}
\jiwon
\end{boxedverbatim}

\begin{tcolorbox}
\jiwon[1]
\end{tcolorbox}

\section{부분적으로 사용하기}

\begin{boxedverbatim}
\jiwon[a] \jiwon[1-2]
\end{boxedverbatim}

\jiwon[a]

\begin{boxedverbatim}
\jiwon[b] \jiwon[13-20]
\end{boxedverbatim}

\jiwon[b]

\begin{boxedverbatim}
\jiwon[c] \jiwon[21-28]
\end{boxedverbatim}

\jiwon[c]

\begin{boxedverbatim}
\jiwon[15]
\end{boxedverbatim}

\jiwon[15]

\begin{boxedverbatim}
\jiwon[23]
\end{boxedverbatim}

\jiwon[23]

\begin{boxedverbatim}
\jiwon[30]
\end{boxedverbatim}

\jiwon[30]

\section{\cs jiwondef}

\begin{boxedverbatim}
\jiwondef{\myjiwon}{7}
\myjiwon
\end{boxedverbatim}

\jiwondef{\myjiwon}{7}
\myjiwon

\section{nopar와 numbers}

\begin{itemize}
\item 패키지 옵션으로 \texttt{[nopar]}를 주면 문단 끝의 \verb|\par|가 붙지 않습니다.
\end{itemize}

\jiwon*[1-3]

\begin{itemize}
\item \texttt{[numbers]} 옵션을 주면 문단에 번호가 붙습니다. 또는
아래와 같이 끄거나 켤 수 있습니다.
\end{itemize}

\begin{boxedverbatim}
\jiwonparnumberon
\end{boxedverbatim}

\jiwonparnumberon
\jiwon[1-3]

\begin{boxedverbatim}
\jiwonparnumberoff
\end{boxedverbatim}

\jiwonparnumberoff
\jiwon[4-5]

\end{document}
